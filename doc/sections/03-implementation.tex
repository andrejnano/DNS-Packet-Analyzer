
\section{Implementation}

\subsection{Project structure}

\dirtree{%
.1 \faFolderOpen \hspace{1pt} /.
.2 \faFolderOpen \hspace{1pt} src/.
.3 \faFile \hspace{1pt} base64.h.
.3 \faFile \hspace{1pt} config.h.
.3 \faFile \hspace{1pt} dns-header.h.
.3 \faFile \hspace{1pt} misc.h.
.3 \faFile \hspace{1pt} parse-dns.h.
.3 \faFile \hspace{1pt} pcap-analysis.h.
.3 \faFile \hspace{1pt} satistics.h.
.3 \faFile \hspace{1pt} syslogger.h.
.3 \faFileCodeO \hspace{1pt} base64.cc.
.3 \faFileCodeO \hspace{1pt} config.cc.
.3 \faFileCodeO \hspace{1pt} dns-export.cc.
.3 \faFileCodeO \hspace{1pt} parse-dns.cc.
.3 \faFileCodeO \hspace{1pt} pcap-analysis.cc.
.3 \faFileCodeO \hspace{1pt} satistics.cc.
.3 \faFileCodeO \hspace{1pt} syslogger.cc.
.2 \faFileTextO \hspace{1pt} dns-export.1.
.2 \faCogs \hspace{1pt} Makefile.
.2 \faFilePdfO \hspace{1pt} manual.pdf.
}

\subsection{Important sections of the code}

Since the primary objective of this project is to process DNS packets, the following functions are the backbone of the whole processing aspect of this application.

\vspace{1cm}
\textbf{Packet processing loop} \textit{(/src/dns-export.cc)}
\begin{lstlisting}[language=C++] 
    // analyze packets returned by the handle
    if (pcap_loop(pcap_handle, -1, pcap_analysis, reinterpret_cast<u_char*>(&link_type)) != 0)
    {
        std::cerr << "packet reading failed" << std::endl;
        return EXIT_FAILURE;
    }
\end{lstlisting}

\vspace{1cm}
\textbf{Parsing of the DNS frame the application layer} \textit{(/src/parse-dns.h)}
\begin{lstlisting}[language=C++] 
   
    /**
     *  @brief Parse DNS frame of the packet
     * 
     *  @param bytes pointer to the packet
     *  @param packet_offset_size current offset in the packet
     */
    void parse_dns(const u_char* bytes, int32_t packet_offset_size);

\end{lstlisting}

\pagebreak

\vspace{1cm}
\textbf{Parsing of DNS Questions \& Answers sections} \textit{(/src/parse-dns.h)}
\begin{lstlisting}[language=C++] 
    /**
     *  @brief Parse a Question in the Questions section on the DNS frame of the packet
     * 
     *  @param bytes pointer to the packet
     *  @param packet_offset_size current offset in the packet
     */
    size_t parse_dns_question(const u_char* bytes, int32_t packet_offset_size);

    /**
     *  @brief Parse an Answer in the Answers section in the DNS frame of the packet
     * 
     *  @param bytes pointer to the packet
     *  @param packet_offset_size current offset in the packet
     */
    int32_t parse_dns_answer(const u_char* bytes, int32_t packet_offset_size);
\end{lstlisting}

\vspace{1cm}
\textbf{Parsing of RDATA field in DNS Answers section} \textit{(/src/parse-dns.h)}
\begin{lstlisting}[language=C++] 
    /**
     *  @brief Parse the Record Data field in the DNS frame 
     * 
     *  @param bytes pointer to the packet
     *  @param packet_offset_size current offset in the packet
     *  @param type DNS RR type 
     *  
     *  @return parsed RDATA
     */
    std::string parse_dns_answer_rdata( const u_char* bytes, int32_t packet_offset_size, uint16_t rdata_length, uint16_t type);
\end{lstlisting}

\vspace{1cm}
\textbf{Parsing of the domain-name type of fields} \textit{(/src/parse-dns.h)}
\begin{lstlisting}[language=C++]
    /**
     *  @brief Parse the label/ptr name field in the DNS frame
     * 
     *  @param bytes pointer to the packet
     *  @param packet_offset_size current offset in the packet
     *  @param name string to which the parsed content will be stored
     *  @param is_pointer_reference flag describing the context in which the name parsing is run
     *  @param depth of the recursive pointer call
     * 
     *  @return updated packet_offset_size
     */
    int32_t parse_dns_name_field(const u_char* bytes, 
                                int32_t packet_offset_size, 
                                std::string &name, 
                                bool is_pointer_reference,
                                uint8_t depth = 0);
\end{lstlisting}

\vspace{1cm}
\textbf{Parsing of the character-string type of fields} \textit{(/src/parse-dns.h)}
\begin{lstlisting}[language=C++]
    /**
     *  @brief Parse one or more <character-string>s and save them into a string
     * 
     *  @param bytes pointer to the packet
     *  @param packet_offset_size current offset in the packet
     *  @param name string to which the parsed content will be stored
     * 
     *  @return updated packet_offset_size
     */
    size_t parse_dns_string(const u_char* bytes, int32_t packet_offset_size, std::string &name);
\end{lstlisting}

\subsection{Runtime flow structure}

\begin{enumerate}
\item declarations, signals, globals setup
\item argument parsing
\item execution mode split
\begin{enumerate}[label=(\alph*)]
\item interface sniffing
\begin{enumerate}
\item create pcap handle
\item set options for sniffing
\item compile \& apply filter ("port 53")
\item create new thread for regular exporting
\item start analyzing packets
\end{enumerate}
\item offline file reading
\begin{enumerate}
\item open the file
\item start analyzing packets
\end{enumerate}
\end{enumerate}
\item exit
\end{enumerate}

\subsection{Limitations}

\begin{enumerate}
\item Application doesn't support TCP fragmentation of packets.
\item Application doesn't support other link layer types than Ethernet \& Linux SLL
\item 
\end{enumerate}