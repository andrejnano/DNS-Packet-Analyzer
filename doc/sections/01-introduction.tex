

\section{Introduction}

\subsection{Description/About}

This software project atempts to solve a university networking course (ISA) assignment
which has the following specifications:

\textit{"Target of the projects is to create application, which will be able to process data of the DNS (Domain Name System)
protocol and will export selected statistics to a central log server using Syslog protocol."}

More detailed description of the assignment is in the following sections.

\subsection{Features}

\begin{itemize}
\item PCAP analysis of static files (*.pcap)
\item PCAP analysis directly from live interfaces
\item L2 layer parsing \textit{Ethernet frame and Linux cooked-mode capture (SLL)}
\item L3 layer parsing \textit{IPv4 and IPv6}
\item L4 layer parsing \textit{TCP and UDP}
\item L7 layer parsing \textbf{DNS}
    \begin{itemize}
    \item \textit{A, NS, CNAME, SOA, PTR, MX, AAAA, TXT, RRSIG, NSEC, DNSKEY, DS}
    \end{itemize}
\item Export of statistics to a specified syslog server
\end{itemize} 
\subsection{Prerequisites}

\begin{itemize}

\item \textbf{Operating Systems}: CentOS7, macOS 10, other Linux/BSD distributions..
\item gcc version 4.8.5+
\item a Syslog server

\end{itemize}


\subsection{Getting started}

\begin{enumerate}
\item Get the latest project version.\\
\texttt{ git clone https://github.com/andrejnano/DNS-EXPORT-PROJECT.git}
\item Change directory to project \\
\texttt{ cd DNS-EXPORT-PROJECT/ }
\item Build the project \\
\texttt{ make }
\item Run the executable \\
\texttt{ dns-export [-r file.pcap] [-i interface] [-s syslog-server] [-t seconds]}
\end{enumerate}

\quad \quad \textit{More usage details are in the section} \textbf{3}