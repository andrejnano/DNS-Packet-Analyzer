

% která bude obsahovat uvedení do problematiky, 
% návrhu aplikace, popis implementace, základní
% informace o programu, návod na použití. 
% V dokumentaci se očekává následující:
%     - titulní strana
%     - obsah
%     - logické strukturování textu
%     - přehled nastudovaných informací z literatury
%     - popis zajímavějších pasáží implementace
%     - použití vytvořených programů a literatura.

\section{Introduction}

\subsection{Description/About}

% Short explanation with problem and quick solution explanation.
% Direction to instantly use by following steps in this section,
% or reading about the design and implementation in sections 
% following this one.

% more details about how to use -> skip to Usage

This software project atempts to solve a university networking course (ISA) assignment
which has following specifications:

"Target of the projects is to create application, which will be able to process data of the DNS (Domain Name System)
protocol and will export selected statistics to a central log server using Syslog protocol."

More detailed description of the assignment is in the following section..

The final state of the solution offered is in my opinion for the most part successfuly complete.

\subsection{Features}

\begin{itemize}
\item PCAP analysis of static files (*.pcap)
\item PCAP analysis directly from live interfaces
\item L2 layer parsing
    \begin{itemize}
    \item Ethernet frame
    \item Linux cooked-mode capture (SLL)
    \end{itemize}
\item L3 layer parsing
    \begin{itemize}
    \item IPv4
    \item IPv6
    \end{itemize}
\item L4 layer parsing
    \begin{itemize}
    \item TCP
    \item UDP
    \end{itemize}
\item L5 layer parsing 
    \begin{itemize}
    \item \textbf{DNS} (\textit{A, NS, CNAME, SOA, PTR, MX, AAAA, TXT, RRSIG, NSEC, DNSKEY, DS resource record types})
    \end{itemize}
\end{itemize} 
\subsection{Prerequisites}

\begin{itemize}

\item \textbf{Operating Systems}
    \begin{itemize}
    \item CentOS7+
    \item Mac OS 10+
    \end{itemize}
\item gcc version 4.8.5+
\item syslog server

\end{itemize}


\subsection{Getting started}

\begin{enumerate}
\item Get the latest project version
\begin{lstlisting}[language=Bash] 
 git clone https://github.com/andrejnano/DNS-EXPORT-PROJECT.git 
\end{lstlisting}
\item Change directory
\begin{lstlisting}[language=Bash] 
cd DNS-EXPORT-PROJECT/
\end{lstlisting}
\item Build the project
\begin{lstlisting}[language=Bash] 
make
\end{lstlisting}
\item Run the application
\begin{lstlisting}[language=Bash] 
dns-export [-r file.pcap] [-i interface] [-s syslog-server] [-t seconds]
\end{lstlisting}
\end{enumerate}